\documentclass[11pt]{article}
\usepackage[utf8]{inputenc} %% codificoación: ñ, diéresis, etc..
\usepackage[spanish]{babel} %% traducir comandos de LaTeX
\usepackage{xcolor} %% abanico amplio de colores
\usepackage{graphicx} % insertar gráficas o imagenes
\usepackage{float} % uso de imagenes flotantes
\usepackage{booktabs} % dar mejor formato de tablas, uso de toprule, midrule,.... 
\usepackage{enumitem} % modificar listas numericas, alphanimericas, etc...
\usepackage{lipsum} % paquete para texto aleatorio
\usepackage{anyfontsize} %tamaño del texto
\usepackage{ragged2e} %para justificar un texto
\usepackage{hyperref} %para referenciar

\setlength{\columnsep}{30pt}

%AMS ---> Formulas matemáticas
\usepackage{amsmath, 
            bm, 
            amsfonts, 
            amssymb}
\usepackage{geometry} % modificación de márgenes
\geometry{
    top = 0.5in,
    inner = 0.5in,
    outer = 0.5in,
    bottom = 0.9in,
}

\definecolor{blueM}{cmyk}{1.0,0.49,0.0,0.47}

\usepackage{fancyhdr} %modificación del encabezado y el pie de página
\pagestyle{fancy}

\fancyhead[c]{} 
\fancyhead[r]{}
\fancyhead[l]{}

\fancyfoot[c]{}
\fancyfoot[l]{}
\fancyfoot[r]{}

\renewcommand{\headrulewidth}{0pt}
\renewcommand{\footrulewidth}{0pt}





\begin{document}

\begin{center}
    \begin{spacing}{2}
    {\bf\fontsize{19}{21}\selectfont\color{azul} Aplicación de la metodología QM - CRISP DM
para la reducción del desperdicio de frutas en el
mercado modelo la Victoria}
    \end{spacing}
\vspace{1mm}
    {\bf (Application of the QM - CRISP DM methodology to reduce fruit waste in the La Victoria model market)}
    
    
    \vspace{2.5mm}
    {\small Carlos Rondan Poma, Universidad Nacional de Ingeniería, Perú\\
Enrique Elera García, Universidad Nacional de Ingeniería, Perú\\
Johana Surco Huancas, Universidad Nacional de Ingeniería, Perú\\
José Gutierrez Saravia, Universidad Nacional de Ingeniería, Perú\\
Luz León Churquipa, Universidad Nacional de Ingeniería, Perú}
    
\end{center}

\begin{spacing}{0.95}
\footnotesize\noindent\itshape Resumen: Perú es el octavo país que produce y genera más desperdicios de plátanos, desperdicio equivalente a
37 millones de dólares a nivel nacional; todo esto sin mencionar los desperdicio que generan las demás frutas.
En este proyecto proponemos aplicar la metodología QM-CRISP DM para determinar las causas de desperdicio
en la cadena de valor del mercado modelo La Victoria. En consecuencia, nuestra propuesta de mejora ataca dos
frentes importantes, el primero es la aplicación de la metodología 5 s en el mercado mayorista y el segundo es
la aplicación de inteligencia artificial para la detección y clasificación del nivel de calidad de los plátanos y
estimación del tiempo de vida útil.
\vspace{-0.2cm}
\begin{center}
    Palabras Clave: Desperdicio de frutas, DMAIC, CRISP-DM, metodología 5s, Aprendizaje profundo
\end{center}
\end{spacing}

\begin{spacing}{0.95}
\footnotesize\noindent\itshape Resumen: Perú es el octavo país que produce y genera más desperdicios de plátanos, desperdicio equivalente a
37 millones de dólares a nivel nacional; todo esto sin mencionar los desperdicio que generan las demás frutas.
En este proyecto proponemos aplicar la metodología QM-CRISP DM para determinar las causas de desperdicio
en la cadena de valor del mercado modelo La Victoria. En consecuencia, nuestra propuesta de mejora ataca dos
frentes importantes, el primero es la aplicación de la metodología 5 s en el mercado mayorista y el segundo es
la aplicación de inteligencia artificial para la detección y clasificación del nivel de calidad de los plátanos y
estimación del tiempo de vida útil.
\vspace{-0.2cm}
\begin{center}
    Palabras Clave: Desperdicio de frutas, DMAIC, CRISP-DM, metodología 5s, Aprendizaje profundo
\end{center}
\end{spacing}
\tableofcontents
\section[Introducción]{Introducción}

\lipsum[1-3]
\lista{
    \item \lipsum[1]
    \item \lipsum[1]
}


\lipsum[1]

\subsection{Planteamiento del problema}

\lipsum[1-3]

\begin{table}[h!]
    \centering
    \begin{tabular}{rrr}
\toprule
 floor\_area &  year\_built &  nueva variable \\
\midrule
    61242.0 &      1942.0 &     118931964.0 \\
   274000.0 &      1955.0 &     535670000.0 \\
   280025.0 &      1951.0 &     546328775.0 \\
    55325.0 &      1980.0 &     109543500.0 \\
    66000.0 &      1985.0 &     131010000.0 \\
   119900.0 &      1956.0 &     234524400.0 \\
    91367.0 &      1982.0 &     181089394.0 \\
    50422.0 &      1947.0 &      98171634.0 \\
   122020.0 &      1929.0 &     235376580.0 \\
   102612.0 &      1979.0 &     203069148.0 \\
    65998.0 &      1979.0 &     130610042.0 \\
   100000.0 &      1927.0 &     192700000.0 \\
   128320.0 &      1960.0 &     251507200.0 \\
   616793.0 &      1955.0 &    1205830315.0 \\
    53000.0 &      1924.0 &     101972000.0 \\
    90045.0 &         NaN &             NaN \\
    74055.0 &      1949.0 &     144333195.0 \\
   128800.0 &      1926.0 &     248068800.0 \\
    91619.0 &      1914.0 &     175358766.0 \\
    53280.0 &      1973.0 &     105121440.0 \\
   217710.0 &      1900.0 &     413649000.0 \\
    68538.0 &      1913.0 &     131113194.0 \\
    90669.0 &      1962.0 &     177892578.0 \\
\bottomrule
\end{tabular}
    \caption{Consumo energético}
    \label{tab:my_label}
\end{table}

\begin{figure}[h!]
    \centering
    \includegraphics[width = 11cm]{figuras/cluster_3D.PNG}]
    \caption{Imagen de un cluster}
\end{figure}

\figura{5cm}{cluster_3D}{Ssoy una imagen de cluster\label{imagen_3D}}
\newpage
La imagen usando un algoritmo de machine learining del tipo K mean (ver figura \ref{imagen_3D})

\section[ANTECEDENTES]{Antecedentes}
\lista{
    \item \lipsum[1]
    \item \lipsum[1]
}
\input{metodologia}
\input{resultados}
\input{conclusiones}


\end{document}
