\documentclass[11pt,letterpaper,twocolumn]{article}
\title{proyecto}
\author{Carlos Rondán}
\date{February 2022}

\usepackage[utf8]{inputenc} %% codificoación: ñ, diéresis, etc..
\usepackage[spanish]{babel} %% traducir comandos de LaTeX
\usepackage{xcolor} %% abanico amplio de colores
\usepackage{graphicx} % insertar gráficas o imagenes
\usepackage{float} % uso de imagenes flotantes
\usepackage{booktabs} % dar mejor formato de tablas, uso de toprule, midrule,.... 
\usepackage{enumitem} % modificar listas numericas, alphanimericas, etc...
\usepackage{lipsum} % paquete para texto aleatorio
\usepackage{anyfontsize} %tamaño del texto
\usepackage{ragged2e} %para justificar un texto
\usepackage{hyperref} %para referenciar

\setlength{\columnsep}{30pt}

%AMS ---> Formulas matemáticas
\usepackage{amsmath, 
            bm, 
            amsfonts, 
            amssymb}
\usepackage{geometry} % modificación de márgenes
\geometry{
    top = 0.5in,
    inner = 0.5in,
    outer = 0.5in,
    bottom = 0.9in,
}

\definecolor{blueM}{cmyk}{1.0,0.49,0.0,0.47}

\usepackage{fancyhdr} %modificación del encabezado y el pie de página
\pagestyle{fancy}

\fancyhead[c]{} 
\fancyhead[r]{}
\fancyhead[l]{}

\fancyfoot[c]{}
\fancyfoot[l]{}
\fancyfoot[r]{}

\renewcommand{\headrulewidth}{0pt}
\renewcommand{\footrulewidth}{0pt}





\begin{document}

\twocolumn[\begin{@twocolumnfalse}


\begin{minipage}{0.15\textwidth}{
    \includegraphics[width=3.5cm]{UNI.png}}
\end{minipage}
\hspace{25pt}
\begin{minipage}{0.75\textwidth}
\vspace{5mm}
    \Large{\textbf{Título del trabajo de investigación}} 
    \vspace{3mm}
    
    \large{\textbf{Rondan, Carlos$^1$}; Arquinigo, luisa$^2$} 
    \vspace{2mm}
    
    \large{\textbf{Asesor 1: Garagay, Luis$^1$} ; Asesor 2: Caja, Ruddy$^2$} \newline
    \fontsize{0.35cm}{0.5cm}\selectfont \textit{Facultad de Ingeniería Industrial y de Sistemas, Universidad Nacional de Ingeniería\newline 
    Av. Tupac Amaru Nro. 210 - Lima, Rimac, Perú}
    \vspace{1mm} 
    
    \today 

\end{minipage}

\small

\vspace{11pt}

\centerline{\rule{0.95\textwidth}{0.4pt}}

\begin{center}
    
    \begin{minipage}{0.9\textwidth}
        % RESUMEN
        \noindent \textbf{Resumen:} {\lipsum[1]}
        \vspace{4mm}
        % PALABRAS CLAVE
        \noindent \textbf{Palabras clave:} Palabra clave 1, palabra clave 2, palabra clave 3, palabra clave 4.
    
    \end{minipage}
    
\end{center}

\centerline{\rule{0.95\textwidth}{0.4pt}}

\vspace{15pt}
\end{@twocolumnfalse}]
\section{Introducción}
\lipsum[1].\par
La forma de presentar una ecuación es la siguiente: 
%%%%%%%%%%%%%%%%%%%%
\begin{equation}
    \label{eq:eg2}
    i \hbar \frac{\partial}{\partial t} \Psi (r,t)= \left[ -\frac{\hbar^{2}}{2m} \nabla^{2} +V(r,y)\right] \Psi (r,t)
\end{equation}
%%%%%%%%%%%%%%%%%%%%
Y la forma de referirse a la ecuación anterior es \ref{eq:eg2} o si quieren que aparezca la palabra ecuación, figura, tabla pueden hacerlo así \autoref{eq:eg2}. \par 

\lipsum[2].\par 
 
\begin{table}[H]
	\begin{center}
		\resizebox{0.45\textwidth}{!}{ %ajusta el tamaño del cuadro
			\begin{tabular}{|c|c|}
			\hline
				Distancia del láser ($\pm 0.1$ cm) & Diámetro del círculo ($\pm 0.1$cm) \\ \hline \hline
				50 & 0.4 \\ \hline
				100 & 0.5 \\ \hline
				150 & 0.6 \\ \hline
				200 & 0.7 \\  \hline
			\end{tabular}
		}
	\end{center}
	\label{cua:1}
	\caption{Valores de distancias y diámetros respectivos}
\end{table}
\lipsum[1-3]

\begin{figure}[ht]
    \centering
    \includegraphics[width=5cm]{NU_Logo.png} 
    \caption{Pie de figura.}
    \label{fig:my_label}
\end{figure}

\lipsum[1]

\begin{table*}[tp]% bp se utilizada para colocar la tabla en la parte inferior de la hoja y el simbolo * es para que una tabla grande no colapse con el texto
\begin{center}
\begin{tabular}{|l|l|l|l|l|}
\hline
Property    & 1995 model & 1995 (with YASP) & This work   & Exper. \\ \hline \hline
Density ($ \,kg m^{-3}$)    & 1099  & 1143 (83) & 1095 (2)     & 1095   \\ \hline
Heat of vaporization ($kJ \,mol^{-1}$)  & 52.87 & 54.89 (0.07)     & 52.42 (0.05) & 52.88  \\ \hline
Diffusion coefficient ($10^{-5} \, cm^{2} s^{-1}$) & 1.1 &  0.68 (0.02)        & 0.88 (0.02)  & 0.8    \\ \hline
Rotational correlation time ($ps$)  & 3.9     &    4.18 (0.01) & 3.50 (0.01)& 5.2    \\ \hline
Thermal expansion coefficient $(10^{-3} \,K^{-1}$)  & 0.91 (0.10) & 0.90 (0.11) & 0.87 (0.09) & 0.928  \\ \hline
\end{tabular}
	\end{center}
	\caption{Physical properties of liquid DMSO at 298 K and 0.1013 MPa}
	\label{a}
\end{table*}

\lipsum[1-8]


\begin{thebibliography}{9}
\bibitem{Bordat} P. Bordat et al. ``An improved dimethyl sulfoxide force field for molecular dynamics simulations'' \textit{J. Chemical Physics} Vol. 374 (2003) p. 201-205.
\end{thebibliography} 


\end{document}
