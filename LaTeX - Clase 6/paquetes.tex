\usepackage[utf8]{inputenc} %% codificoación: ñ, diéresis, etc..
\usepackage[spanish]{babel} %% traducir comandos de LaTeX
\usepackage{xcolor} %% abanico amplio de colores
\usepackage{graphicx} % insertar gráficas o imagenes
\usepackage{float} % uso de imagenes flotantes
\usepackage{booktabs} % dar mejor formato de tablas, uso de toprule, midrule,.... 
\usepackage{enumitem} % modificar listas numericas, alphanimericas, etc...
\usepackage{lipsum} % paquete para texto aleatorio
\usepackage{anyfontsize} %tamaño del texto
\usepackage{ragged2e} %para justificar un texto
\usepackage{hyperref} %para referenciar

\setlength{\columnsep}{30pt}

%AMS ---> Formulas matemáticas
\usepackage{amsmath, 
            bm, 
            amsfonts, 
            amssymb}
\usepackage{geometry} % modificación de márgenes
\geometry{
    top = 0.5in,
    inner = 0.5in,
    outer = 0.5in,
    bottom = 0.9in,
}

\definecolor{blueM}{cmyk}{1.0,0.49,0.0,0.47}

\usepackage{fancyhdr} %modificación del encabezado y el pie de página
\pagestyle{fancy}

\fancyhead[c]{} 
\fancyhead[r]{}
\fancyhead[l]{}

\fancyfoot[c]{}
\fancyfoot[l]{}
\fancyfoot[r]{}

\renewcommand{\headrulewidth}{0pt}
\renewcommand{\footrulewidth}{0pt}



