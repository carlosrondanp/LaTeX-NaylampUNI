\usepackage[utf8]{inputenc} %%  codificoación: ñ, diéresis, etc..
\usepackage[T1]{fontenc} % Sin esto tendíamos que poner \'a
\usepackage[spanish]{babel} %% traducir comandos de LaTeX
\usepackage{xcolor} %% abanico amplio de colores
\usepackage{graphicx} % insertar gráficas o imagenes
\usepackage{float} % uso de imagenes flotantes
\usepackage{array} %para poder definir el ancho de las columnas en mi tabla de la siguiente manera: \begin{tabular}{|m{2cm}|m{5cm}}
\usepackage{booktabs} % dar mejor formato de tablas, uso de toprule, midrule,.... 
\usepackage{enumitem} % modificar listas numericas, alphanimericas, etc...
\usepackage{lipsum} % paquete para texto aleatorio
\usepackage{anyfontsize} %tamaño del texto
\usepackage{ragged2e} %para justificar un texto
\usepackage{hyperref} %para referenciar
\setlength{\columnsep}{30pt}


\hypersetup{
colorlinks = true, % si pones true elimina las cajas
linkcolor = rojo,
citecolor = azul 
}

%AMS ---> Formulas matemáticas
\usepackage{amsmath, 
            bm, 
            amsfonts, 
            amssymb}
\usepackage{geometry} % modificación de márgenes
\geometry{
    top = 0.8in,
    inner = 0.5in,
    outer = 0.5in,
    bottom = 0.9in,
}

\definecolor{blueM}{cmyk}{1.0,0.49,0.0,0.47}

\usepackage{fancyhdr} %modificación del encabezado y el pie de página
\pagestyle{fancy}

\fancyhead[c]{\small\color{azul} C. Rondan et. al} 
\fancyhead[r]{\color{gray} $|$ pag. N° \small\thepage}
\fancyhead[l]{}

\fancyfoot[c]{}
\fancyfoot[l]{}
\fancyfoot[r]{}

\renewcommand{\headrulewidth}{0pt}
\renewcommand{\footrulewidth}{0pt}

\definecolor{azul}{rgb}{0.11,0.38,0.65} %definir colores --> color{azul}
\definecolor{gris}{gray}{0.2}
\definecolor{rojo}{rgb}{0.65,0.12,0.06}


% Titlesec sirve para modificar el formato de los capítulos, secciones, subsecciones, etc..
\usepackage{titlesec}
\titleformat
{\section} %elegimos el nivel: captíulo, section, subsection ....
[hang]
{\Large\bf} % modifica las características que tendrá la siguiente línea de código
{\thesection.  } % Se puede poner algo antes del texto, por ejemplo la numeración 1.  o tal vez poner 'Capítulo' ...
{0pt} % Modifica el espacio izquierdo del texto, |INTRODUCCIÓN,  |    INTRODUCCIÓN 
{\Large} %modifica el texto, por ejemplo: que sea grande, cursiva, color azul, etc...s
[\vspace{-0.4cm}\rule{0.47\textwidth}{0.2pt}] 

\titleformat
{\subsection}
[hang]
{\normalsize\bf}
{}
{0pt}
{\itshape\large}
{}

\newcommand{\Color}[1]{\hypersetup{linkcolor=#1}\color{#1}} % lo usaremos para chancar los colores y priorizar la coniguración de lo hipervinculos

\usepackage{tocloft}
%%%%%%%%%% ESPACIO ENTRE INDICE, INDICE DE FIGURAS E INDICE DE CUADROS
\setlength{\cftbeforetoctitleskip}{.5cm} % espacio antes de la tabla de TOC
\setlength{\cftaftertoctitleskip}{0cm} % espacio despues del TOC

\setlength{\cftbeforeloftitleskip}{.5cm} % espacio antes de la tabla de LOF
\setlength{\cftafterloftitleskip}{0cm} % espacio despues del LOC

\setlength{\cftbeforelottitleskip}{.5cm} % espacio antes de la tabla de LOF
\setlength{\cftafterlottitleskip}{0cm} % espacio despues del LOC

%%%%%%%%%% CONFIGURACIÓN DE LETRA DE INDICE, INDICE DE FIGURAS Y,..
\renewcommand{\cfttoctitlefont}{\bf\color{azul}\hfil\Large\sffamily} % configuración del titulo del TOC
% cambiar lo que está en {} por toc, lof o lot---> \cft{toc}titlefont
\renewcommand{\cftloftitlefont}{\bf\color{azul}\hfil\Large\sffamily} 
\renewcommand{\cftlottitlefont}{\bf\color{azul}\hfil\Large\sffamily} 


%%%%%%%%%% TIPO DE LETRA DE SECCIÓN, SUBSECCIÓN, FIGURA, TABLA
%%% Modificar texto, incluye tamaño, color, tipo de texto ,etc...
\renewcommand{\cftsecfont}{\bf\Color{black}} %modificar el texto TOC de section
\renewcommand{\cftsubsecfont}{\Color{gris}}  %modificar el texto TOC de subsection

\renewcommand\cftfigfont{\bf\Color{black}}

\renewcommand\cfttabfont{\bf\Color{black}}

%%%%%%%%%%% CONFIGURACIÓN DE LA NUMERACIÓN
\renewcommand{\cftsecpagefont}{\color{gris}} % modificar el color de la numeración de la TOC en section
\renewcommand{\cftsubsecpagefont}{\color{gris}} %modificar el color de la numeración de la TOC en subsection

\renewcommand{\cftfigpagefont}{\color{gris}}

\renewcommand{\cfttabpagefont}{\color{gris}}


%%%%%%%%%%%% 
\setlength{\cftbeforesecskip}{0.15cm} %espacio antes de cada sección (en la tabla)
\setlength{\cftbeforesubsecskip}{0.15cm} %espacio antes de cada subsección (en la tabla)
\setlength{\cftbeforesubsubsecskip}{0.15cm}%espacio antes de cada subsubsección 

\setlength{\cftbeforefigskip}{0.15cm}

\setlength{\cftbeforetabskip}{0.15cm}

%%%%%%%%%%% AÑADIR SECCION, SUBSECCION, FIGURA, TABLA EN INDICE

\renewcommand\cftsecpresnum{\sffamily Sección }
\setlength{\cftsecnumwidth}{2.0cm} %distancia de las palabras con la numeración (section)
\renewcommand\cftsubsecpresnum{\sffamily Sub sección }
\setlength{\cftsubsecnumwidth}{2.5cm} %distancia de la letra al numero 1. (espacio) nombre

\renewcommand\cftfigpresnum{\sffamily Figura }
\setlength{\cftfignumwidth}{1.8cm}

\renewcommand\cfttabpresnum{\sffamily Figura }
\setlength{\cfttabnumwidth}{1.8cm}


%%%%%%%%%%%%%% PUNTOS EN EL INDICE , FIGURAS Y EN TABLAS
\renewcommand\cftsecleader{\cftdotfill{2}}
\renewcommand\cftsubsecleader{\cftdotfill{5}}

\renewcommand\cftfigleader{\cftdotfill{5}}

\renewcommand\cfttableader{\cftdotfill{5}}

%%%%%%%%%%%%%%%%




\addto\captionsspanish{\renewcommand{\figurename}{\small\sffamily \textcolor{azul}{\bfseries Fig. N°}}}  %cambia formato del caption   % le quité \itshape
\addto\captionsspanish{\renewcommand{\tablename}{\small\sffamily \textcolor{azul}{\bfseries Tabla N°}}}  %cambia formato del caption



\usepackage[square,numbers]{natbib} %[square,numbers] , round
\bibliographystyle{unsrtnat} %abbrvnat