\usepackage[utf8]{inputenc} % codificoación: ñ, diéresis, etc..
\usepackage[T1]{fontenc} % sin esto tendríamos que poner \'a
\usepackage{lipsum} % texto aleatorio
\usepackage[spanish]{babel}
\title{Articulo}
\author{Carlos Rondán}
\date{\today}

%AMS: American Mathematical society --> uso de fórmulas matemáticas
\usepackage{amsmath, 
            bm, 
            amsfonts, 
            amssymb}

% geometry: configuración de los margenes de tu artículo
\usepackage[left = 2.5cm,  %margen izquierdo
            right = 1.5cm, %margen derecho
            top = 2cm,     %margen superior
            bottom = 2cm]  %margen inferior
{geometry} 
            
            
\usepackage{anyfontsize} %tamaño del texto dentro de un entorno

\linespread{0.95} %interlineado general
\usepackage{setspace} % interlineado dentro del entorno \begin{spacing}{#}


\usepackage{enumitem} %Listas enumeradas

\usepackage{booktabs} 

\usepackage{graphicx} % insertar imagenes
\usepackage{float} % ubicación de imagenes flotantes

\usepackage{fancyhdr}
\pagestyle{fancy}
\fancyhead[c]{\small \textcolor{azul}{C. Rondan et al.}}
\fancyhead[r]{{| N° Pag.\thepage}}
\fancyhead[l]{}
\fancyfoot[c]{Universidad Nacional de Ingeniería}
\fancyfoot[l]{}
\fancyfoot[r]{}
\renewcommand{\headrulewidth}{0pt}
\renewcommand{\footrulewidth}{0pt}

\usepackage{xcolor}
\definecolor{azul}{rgb}{0.11,0.38,0.65}
\definecolor{gris}{gray}{0.875}
\definecolor{rojo}{rgb}{0.65,0.12,0.06}


\usepackage{titlesec}
\titleformat
{\section}
[hang]
{\normalsize\bf}
{}
{0pt}
{\Large}
[\vspace{-0.4cm}\rule{\textwidth}{0.3pt}]

\titleformat
{\subsection}
[hang]
{\normalsize\bf}
{}
{0pt}
{\itshape\Large}
{}




%%%%%%%%%%%% FUNCIONES %%%%%%%%%%%%%%%%%%%%
\newcommand{\lista}[1]{
\begin{enumerate}[itemsep = -3pt, leftmargin = 15pt, label = \bf\alph*)]
    #1
\end{enumerate}
}
\newcommand{\figura}[3]{
\begin{figure}[h!]
    \centering
    \includegraphics[width = #1]{figuras/#2}]
    \caption{#3}
\end{figure}
}
%%%%%%%%%%%%%%%%%%%%%%%%%%%%%%%%%%%%%%%%%%%%%%%

\usepackage{tocloft}
\setlength{\cftbeforetoctitleskip}{1cm} %antes
\setlength{\cftaftertoctitleskip}{-1cm} % despues

\renewcommand{\cfttoctitlefont}{\bf\color{azul}\hfil\scshape} % configuración del titulo del TOC
% cambiar lo que está en {}---> \cft{toc}titlefont
% por toc (talba de contenidos), lof (lista de figuras), lot (lista de tablas)



\renewcommand{\cftsecfont}{\bf\color{rojo}} %modificar el texto TOC de section
\renewcommand{\cftsubsecfont}{}  %modificar el texto TOC de subsection



\renewcommand{\cftsecpagefont}{\color{rojo}} % modificar el color de la numeración de la TOC en section
\renewcommand{\cftsubsecpagefont}{}% modificar el color de la numeración de la TOC en subsection