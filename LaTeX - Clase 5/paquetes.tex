\usepackage[utf8]{inputenc} % codificoación: ñ, diéresis, etc..
\usepackage[T1]{fontenc} % sin esto tendríamos que poner \'a
\usepackage{lipsum} % texto aleatorio
\usepackage[spanish]{babel}
\title{Articulo}
\author{Carlos Rondán}
\date{\today}

%AMS: American Mathematical society --> uso de fórmulas matemáticas
\usepackage{amsmath, 
            bm, 
            amsfonts, 
            amssymb}

% geometry: configuración de los margenes de tu artículo
\usepackage[left = 2.5cm,  %margen izquierdo
            right = 1.5cm, %margen derecho
            top = 2cm,     %margen superior
            bottom = 2cm]  %margen inferior
{geometry} 
                 
\usepackage{anyfontsize} %tamaño del texto dentro de un entorno

\linespread{0.95} %interlineado general
\usepackage{setspace} % interlineado dentro del entorno \begin{spacing}{#}

\usepackage{enumitem} %Listas enumeradas

\usepackage{booktabs} %Hace las tablas más bonitas. Pueden usar los siguientes comandos:
%\toprule,
%\midrule
%\botomrule
%\cmidrule{m-n}mn
%\cmidrule(lr){m-n}

\usepackage{graphicx} % insertar imagenes
\usepackage{float} % ubicación de imagenes flotantes


%Modificar los encabezados o pies de págnia
\usepackage{fancyhdr} 
\pagestyle{fancy}
\fancyhead[c]{\small \textcolor{azul}{C. Rondan et al.}} 
\fancyhead[r]{{| N° Pag.\thepage}}
\fancyhead[l]{}
\fancyfoot[c]{Universidad Nacional de Ingeniería}
\fancyfoot[l]{}
\fancyfoot[r]{}
\renewcommand{\headrulewidth}{0pt}
\renewcommand{\footrulewidth}{0pt}

\usepackage{xcolor} %Paquete que nos permite usar un amplio abanico de colores
\definecolor{azul}{rgb}{0.11,0.38,0.65} %definir colores --> color{azul}
\definecolor{gris}{gray}{0.875}
\definecolor{rojo}{rgb}{0.65,0.12,0.06}

% Titlesec sirve para modificar el formato de los capítulos, secciones, subsecciones, etc..
\usepackage{titlesec}
\titleformat
{\section} %elegimos el nivel: captíulo, section, subsection ....
[hang]
{\Large\bf} % modifica las características que tendrá la siguiente línea de código
{\thesection.  } % Se puede poner algo antes del texto, por ejemplo la numeración 1.  o tal vez poner 'Capítulo' ...
{0pt} % Modifica el espacio izquierdo del texto, |INTRODUCCIÓN,  |    INTRODUCCIÓN 
{\Large} %modifica el texto, por ejemplo: que sea grande, cursiva, color azul, etc...s
[\vspace{-0.4cm}\rule{\textwidth}{0.3pt}] 

\titleformat
{\subsection}
[hang]
{\normalsize\bf}
{}
{0pt}
{\itshape\large}
{}

% MODIFICAR FUENTES, COLORES, ESAPCIOS DE LAS TABLAS DE CONTENIDO: TOC, LOF, LOT
%TOC: Tabla de contenidos
%LOF: Lista de figuras
%LOT: Lista de tablas
\usepackage{tocloft}
\setlength{\cftbeforetoctitleskip}{1cm} % espacio antes de la tabla de TOC
\setlength{\cftaftertoctitleskip}{-1cm} % espacio despues del TOC

\renewcommand{\cfttoctitlefont}{\bf\color{azul}\hfil\scshape} % configuración del titulo del TOC
% cambiar lo que está en {} por toc, lof o lot---> \cft{toc}titlefont

%%% Modificar texto, incluye tamaño, color, tipo de texto ,etc...
\renewcommand{\cftsecfont}{\bf\color{rojo}} %modificar el texto TOC de section
\renewcommand{\cftsubsecfont}{}  %modificar el texto TOC de subsection


%%% Modificar
\renewcommand{\cftsecpagefont}{\color{rojo}} % modificar el color de la numeración de la TOC en section
\renewcommand{\cftsubsecpagefont}{}% modificar el color de la numeración de la TOC en subsection


%%%%%%%%%%%% FUNCIONES %%%%%%%%%%%%%%%%%%%%
\newcommand{\lista}[1]{
	\begin{enumerate}[itemsep = -3pt, leftmargin = 15pt, label = \bf\alph*)]
		#1
	\end{enumerate}
}
\newcommand{\figura}[3]{
	\begin{figure}[h!]
		\centering
		\includegraphics[width = #1]{figuras/#2}]
		\caption{#3}
	\end{figure}
}
%%%%%%%%%%%%%%%%%%%%%%%%%%%%%%%%%%%%%%%%%%%%%%%